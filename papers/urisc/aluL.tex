The components can be combined to specify exactly how the ALU should behave.
This could, if desired, be verified against the mathematics of of a subset of integers.
\begin{verbatim}
(*              ALU.L
                =====

The definition of the ALU as a whole.

3/1/89 sal: TTL part only; not the PALS
*)

(* The carry look-ahead unit *)
(* built out of three TTL devices *)

val carrylookahead #( g0,g1,g2,g3,g4,g5,g6,g7,
                        p0,p1,p2,p3,p4,p5,p6,p7,
                        c,c3,c7,c11,c15,c19,c23,c27,carry_out)=

        SN74F182 #(g0,g1,g2,g3,p0,p1,p2,p3,c,c3,c7,c11,G0,P0)
        /\
        SN74F182 #(g4,g5,g6,g7,p4,p5,p6,p7,c15,c19,c23,c27,G1,P1)
        /\
        SN74F182 #(G0,G1,true,true,P0,P1,true,true,c,c15,carry_out,
                        false,true,true);


(* the TTL portion of the ALU *)
(* eight four bit slices connected together via the carry generator *)
(* a= Input a
   b= Input b
   c= carry in
   s2-s0= control signals
   c=carry out
   f=evaluated function
    *)


val ttlALU # (a b c s2 s1 s0 carry_out f)=
        (a7,a6,a5,a4,a3,a2,a1,a0)==split a /\
        (b7,b6,b5,b4,b3,b2,b1,b0)==split b /\
        carrylookahead #( g0,g1,g2,g3,g4,g5,g6,g7,
                        p0,p1,p2,p3,p4,p5,p6,p7,
                        c,c3,c7,c11,c15,c19,c23,c27,carry_out) /\
        SN74F381#(a0,b0,c,p0,g0,s2,s1,s0,f0) /\
        SN74F381#(a1,b1,c3,p1,g1,s2,s1,s0,f1) /\
        SN74F381#(a2,b2,c7,p2,g2,s2,s1,s0,f2) /\
        SN74F381#(a3,b3,c11,p3,g3,s2,s1,s0,f3) /\
        SN74F381#(a4,b4,c15,p4,g4,s2,s1,s0,f4) /\
        SN74F381#(a5,b5,c19,p5,g5,s2,s1,s0,f5) /\
        SN74F381#(a6,b6,c23,p6,g6,s2,s1,s0,f6) /\
        SN74F381#(a7,b7,c27,p7,g7,s2,s1,s0,f7) /\
        split f=(f7,f6,f5,f4,f3,f2,f1,f0);

\end{verbatim}
