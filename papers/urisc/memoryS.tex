The memory of the computer can be specified with some difficulty.
RAM is described by a curried function.
A number of boolean functions decode addresses and generate the halt signal.
The read function returns either a RAM location's contents, the accumulator or
a condition flag.
The write function is more complex, being able to update the entire machine state.
\begin{verbatim}
(*	memory.SIM	v1.7 	5/24/89 *)
(*	==========	====	======= *)
(* Simulated memory functions *)         
(* functional memory specification: very memory inefficient *)

	     fun RAM (d:Int32) mem (a:Int15) aa= 
	          (* taken from Lambda's examples *)
                if a=aa then d else mem aa; 
                
           (*reset state -not quite true*)
          fun RAM0 (_:Int15)=Zero32; 
           
          (* A function to test if an address is that of ram or not 
                i.e. returns True iff  bit 14 is True *)

          fun ram_address a=
                addressBit14  a;

          (* A function to test if an address references the alu-
                i.e is in the range of addresses 16-31
                -or equivalently bit 4 =True and it is not a ram address *)

          fun alu_address a=
                ~(ram_address  a) && (addressBit4  a);


          (*  a function to check if an address is valid for a read.
          a boolean value will be returned, true if an address is valid,
          false otherwise.  This is  used to  intercept reads on write only
          registers.
          Can be implemented as a PAL equation *)

          (* A function to read memory.  Will either  return the  data at a
          RAM 
          location or the contents of a memory mapped register*)

          (* invalid === the address is <=7 *)


          fun valid_address a=
                (ram_address  a) |||
                (alu_address  a) |||
                (addressBit3  a);

          exception can't_read; 

          fun read state a= 
                if ram_address a then get_mem state a 
                else if addressBit4 a then 
                  Expand ( if addressBit1 a then 
                   if addressBit0 a then get_carry (get_alu state) 
                                     else get_overflow (get_alu state) 
                    else if addressBit0 a then get_negative (get_alu state)

                         else get_zero ( get_alu state)) 
                else  
		 if addressBit3 a then get_acc (get_alu state)
		 else raise can't_read; 

          (* Write 
                This function takes a  state, an  address and  a data
                item. It  will then  use this item to update the RISC
                state. The program counter will always be  adjusted -
                normally being  incremented, but after a write to the
                PC then it will be  changed  to  the  supplied value.
                Other internal registers - X,Y,skip,Acc and Carry can
                also  be  written  to.  Writing  to   the  other  ALU
                addresses causes  the ALU  to be activated to perform
                an operation upon the Accumulator,  the carry and the
                data supplied.
                Writes to  a ram  address cause the data to be stored
                at the specifed location. *)

          fun write state a data=
           if a=PC orelse (a=PLUS4 PC) then     (* program counter*)
                ({       mem=get_mem state,
                        exstate={
                                pc=Truncate15 data,
                                x=get_x (get_ex state),
                                y=get_y (get_ex state),
                                halt=get_halt (get_ex state)},
                        alustate=get_alu state}:State)
           else
           if a=SKIP orelse (a=PLUS4 SKIP) then
                {       mem=get_mem state,
                        exstate= {pc=
				(if dataBit0 data then
					S15 (get_pc (get_ex state))
				else get_pc (get_ex state)),
                                x=get_x (get_ex state),
                                y=get_y (get_ex state),
                                halt=get_halt (get_ex state)},
                        alustate=get_alu state} 
           else
           if a=X orelse a=PLUS4 X then
                {       mem=get_mem state,
                        exstate= {
                                pc=get_pc (get_ex state),
                                x=Truncate15 data,
                                y=get_y (get_ex state),
                                halt=get_halt (get_ex state)},
                        alustate=get_alu state}
           else
           if a=Y orelse a=PLUS4 Y then
                {       mem=get_mem state,
                        exstate= {
                                pc=get_pc (get_ex state),
                                x=get_x (get_ex state),
                                y=Truncate15 data,
                                halt=get_halt (get_ex state)},
                        alustate=get_alu state}
           else
           if  ram_address a then
                {       mem=RAM data (get_mem state) a,
                        exstate=get_ex state,
                        alustate=get_alu state}
           else
           if alu_address a then
                {       mem = get_mem state,
                        exstate= get_ex state,
                        alustate= alu (get_alu state) data a
                        }
           else
                (* else write to carry flag *)
                (* warning: the state of the N,Z,V flags can't be predicted
                        after this operation: an ADD 0 operation should be
                        performed to reevaluate the other flags-
			an action which'll clear the carry *)
                {       mem = get_mem state,
			exstate=get_ex state,
                        alustate= let
				val { acc=_,z=z,n=n,v=v,carry=c}=
					alu (get_alu state) data a
				in
				{ acc=get_acc (get_alu state),
                                        z=z,
                                        n=n,
                                        v=v,
                                        carry=c}
				end
                        }; 
\end{verbatim}
