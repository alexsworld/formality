\chapter{Monitor Programs}
These are the two programs used to provide the monitor service on the
APM.
\section{IMP Main Monitor}
This main monitor is written in IMP.
It is derived from the CS2 M6809 Monitor, which forms about a third of the code.
The remaining 1400 lines are specific to the Ultimate RISC.

\begin{verbatim}

! Urisc Monitor Program
! 14/3/89 
! Steve Loughran
! based upon M6809 monitor (talks to board through PIA)
! for 6809 control + user interface

{------------------------------------------------------------------------------}
{                                                                              }
{   M6809 Monitor                                                              }
{                                                                              }
{   Version 2.0 DJR 1-Oct-1987                                                 }
{   Version 1.0 DWB 30-Jan-1987                                                }
{                                                                              }
{------------------------------------------------------------------------------}

%option "-low -nons"

%include "inc:util.imp"

%begin

%conststring(31) downloadmonitor = "cs2:m6809mon.obj"
%conststring(31) downloadpia="sal:pia5.obj"
%conststring(31) version="v 5.0"
%constbyte stringsize=80
%constinteger timelimit=90000

%routine %spec printmon(%string(255) s)
%routine %spec checkm6809
%routine %spec haltm6809
%routine %spec nominate infile(%string(80) file)
%routine %spec nominate outfile(%string(80) file)
%integer %fn %spec get file char
%routine %spec close files
%string(80)%fn %spec strip(%string(80) s)
%predicate %spec intparam(%string(80) p, %integername ip)
%predicate %spec getfromtofile(%integername from, to, %string(*)%name file)
%predicate %spec getadval(%integername ad, val)
%predicate %spec getfromtoval(%integername from, to, val)
%predicate %spec getfilead(%string(*)%name file, %integername ad)
%routine %spec print help
%routine %spec memory test(%INTEGER from,to,value,%integername count)
%routine %spec soak test(%INTEGER reps,failures)
%routine %spec dump area(%integer from, to, %string(80) file)
%routine %spec load object file(%string(80) file)
%routine %spec load file(%string(80) file, %integer from)
%routine %spec transfer to file(%integer from, to, %string(80) file)
%routine %spec store byte(%integer ad, val)
%routine %spec store area(%integer from, to, val)
%string(2)%fn %spec hexbyte(%integer i)
%string(4)%fn %spec hexword(%integer i)
%routine %spec zerostore
%routinespec run 6809 program

!and urisc forward specifications

%routinespec transmit
%routinespec clock cpu
%ROUTINESPEC clock shadow
%ROUTINESPEC get state
%PREDICATESPEC get registers
%PREDICATESPEC put registers
%ROUTINESPEC print registers
%ROUTINESPEC initialise URISC
%ROUTINESPEC update state
%ROUTINESPEC update
%ROUTINESPEC wyse scroll region(%byte t,b)
%PREDICATESPEC get bool(%string(40) p,%bytename bool)
%PREDICATESPEC get reg n val(%STRING(*)%NAME reg,%INTEGERNAME ad)
%ROUTINESPEC set register(%STRING(*)%NAME r,%INTEGER val)
%PREDICATESPEC store registers
%PREDICATESPEC urisc write(%INTEGER addr,data)
%PREDICATESPEC urisc read(%INTEGER addr,%INTEGERNAME data)
%ROUTINESPEC urisc download(%string(80) file)
%ROUTINESPEC urisc execute(%integer n)
%routinespec full speed

%constinteger memtop = 16_10000
%constinteger periphbot = 16_C000
%constinteger periphtop = 16_E000
%constinteger goadh = 16_E0
%constinteger goadl = 16_00

@16_F00000 %byteintegerarray memory(0:memtop-1)   {M6809 memory}
@16_F0C000 %byteinteger control latch             {M6809 control bits}

@16_F0BFE0 %byteinteger fstxs         {frame store transmit status byte}
@16_F0BFE1 %byteinteger fstx          {frame store mask byte transmit buffer}
@16_F0BFE2 %byteinteger fscol         {frame store colour register}
@16_F0BFE3 %byteinteger fsenb         {frame store enable register}
{6809 board only allows byte accesses!}
@16_F0BFE4 %byteinteger fsxh          {frame store X address high byte}
@16_F0BFE5 %byteinteger fsxl          {frame store X address low byte}
@16_F0BFE6 %byteinteger fsyh          {frame store Y address high byte}
@16_F0BFE7 %byteinteger fsyl          {frame store Y address low byte}
%short %integer fsx, fsy

@16_F0BFE8 %byteinteger cmtxs         {colour map transmit status byte}
@16_F0BFE9 %byteinteger cmtx          {colour map colour number transmit buffer}
@16_F0BFEA %byteinteger cmrgbh        {colour map colour high byte}
@16_F0BFEB %byteinteger cmrgbl        {colour map colour low byte}

@16_F0BFF0 %byteinteger vdutxs        {vdu transmit status byte}
@16_F0BFF1 %byteinteger vdutx         {vdu transmit buffer}
@16_F0BFF2 %byteinteger keyrxs        {keyboard receive status byte}
@16_F0BFF3 %byteinteger keyrx         {keyboard receive buffer}
%constinteger keybufl = 511
%byteintegerarray keybuf(0:keybufl)
%integer keybufn, keybufinp, keybufoutp

@16_F0BFF4 %byteinteger filtxs        {file transmit status byte}
@16_F0BFF5 %byteinteger filtx         {file transmit buffer}
@16_F0BFF6 %byteinteger filrxs        {file receive status byte}
@16_F0BFF7 %byteinteger filrx         {file receive buffer}

@16_F0BFFC %byteinteger regtxs        {m6809 register transmit status byte}
@16_F0BFFD %byteinteger regtx         {m6809 register transmit buffer}
@16_F0BFFE %byteinteger mrun          {used to halt m6809}
@16_F0BFFF %byteinteger iflg          {used to flag cause of halt}


%constinteger bel = 7, ctrl c = 3, ctrl z = 26

%constinteger area = 1, byte = 2, continue = 3, init = 4, dump = 5, go6809 = 6,
  help = 7, infile = 8, load6809 = 9, memtest = 10, object = 11,
  outfile = 12, quit = 13, c registers = 14, reset6809 = 15, step = 16,
  transfer = 17, zero = 18,
  c freerun=19,
    c clock=20,c go=21,c loadstore=22,c mode=23, c dclk=24,
    c sdo=25, c reset=26,c transmit=27,c get=28,c put=29,
    c set=30,c load=31,c store=32,c write=33,c read=34,
    c soak=35, c download=36,c execute=37,c full speed=38

%constinteger maxcom = 38

%conststring(10)%array coms(1:maxcom) = "area", "byte", "continue", "init",
  "dump", "run6809", "help", "infile", "ld6809", "memtest", "object",
  "outfile", "quit", "registers", "reset6809", "step", "transfer", "zero",
  "freerun","clock","go","l/s","mode","dclk","sdo","reset","tx",
  "get","put","set","load","store","write","read","soak","download",
  "execute","spin"

%string(80) comline, command, params, head, tail, file, filein, fileout
%integer intcom,value, i, c, n, from, to, ad, val, framestore, progstarth, progstartl
%switch com(1:maxcom)

{***********************************************************************}
! URISC definitions

! The registers


@16_F00020 %bytearray register values(1:21)
%conststring(15) register file="sal:registers"
%string(stringsize)%array register names(1:21);
%bytearray register subscripts(1:21);
%byte registers
@16_F0001E %byte pia regs
@16_F0001F %byte pia dir

! plus the names of address & data registers

%conststring(stringsize) MAR="mar",MDR="data",PC="pc"

! state constants

%CONSTBYTE  halted state=0
%CONSTBYTE  waiting state=1
%constbyte loaded state=2


!and the register banks

%BYTE register banks
%STRING(stringsize)%ARRAY bank names(1:21)
%BYTEARRAY bank sizes(1:21)

%string(stringsize) reg

! The states

%string(stringsize)%array state names(0:31)
%conststring(10) state file="sal:states"

!PIA connection

! addresses shared with 6809 virtual PIA

@16_F00000 %byteinteger addr
@16_F00001 %byteinteger data
@16_F00002 %bytearray pia copy(0:3)
@16_F00005 %byte  times reg
@16_F00007 %byte  pause reg
@16_F00008 %byte val1
@16_F00009 %byte val2

!record of old PIA state

%BYTE opa=0

! set to true once the PIA has failed

%INTEGER PIAFAIL=0

! a few variables for the monitor

! the shared printing address (default value)
%integer print addr=16_3FFF

%byte state,halt,sdi,sdo,freerun,clock,
   go,load,loadstore,mode,dclk,reset,flag,changed


%constbyte dataA=0,ctrlA=1,dataB=2,ctrlB=3

! and the escape character for the screen display
%string(1) escape=tostring(27)

! and now the monitor program

  %on %event 0, 9,11 %start
    %if event_event = 9 %then %start    {ctrl-z}
      printsymbol(bel)
      printmon("input ended")
      closefiles
      %stop
    %finish
    %if event_event=0 %start
    {event 0}
    %if event_sub = 0 %then %start    {ctrl-c}
      haltm6809
      printsymbol(bel)
      printmon("break")
      ->nextcommand
    %finish %else %c
    %if event_sub = 1 %then %start    {ctrl-y}
      closefiles
      printsymbol(bel)
      printmon("terminated")
      %stop
    %finish %else %c
    closefiles %and %signal %event 0, event_sub, event_extra
    %finish %elsec
        %if event_sub=0 %thenc
            printmon(" '0' not passed through SSR chain") %c
        %elsec
        %if event_sub=1 %thenc
            printmon(" '1' not passed through SSR chain") %c
        %elsec
        %if event_sub=3 %thenc
            printmon("Virtual PIA Inactive") %and PIAFAIL=1 %c
        %elsec
        %if event_sub=4 %thenc
            printmon("Attempted SSR blowback!")
%FINISH


  select input(0)
  select output(0)

  {check M6809 board exists}
  checkm6809

  haltm6809

  zerostore
  load object file(downloadpia)
  run 6809 program
  initialise urisc
  printmon("Ultimate RISC and M6809 Monitors")
  printmon(version)

  

  filein = ""
  fileout = ""


nextcommand:
 update

  prompt( "mon: " )

  readline(comline) %until comline # ""
  to lower(comline)

  comline = strip(comline)    {clean up spaces}
  %unless comline -> command.(" ").params %then command=comline %and params = ""

  %if command = "" %then ->nextcommand

  intcom = 0    {no command recognised yet}
  {consider all command names}
  %for i = 1, 1, maxcom %cycle
    %if coms(i) -> head.(command).tail %then %start
      %if head = "" %then %start    {i.e. command name not amongst parameters}
        %if intcom # 0 %then printmon("ambiguous command : `". %c
          command."'") %and ->nextcommand
        intcom = i
      %finish
   %finish
  %repeat
  %if intcom = 0 %then printmon("unknown command : `".command."'") %c
    %and ->nextcommand

  ->com(intcom)

com(area):
  %if getfromtoval(from, to, val) %then store area(from, to, val)
  ->nextcommand

com(byte):
  %if getadval(ad, val) %then store byte(ad, val)
  ->nextcommand

com(continue):
  %if params # "" %then printmon("parameters : `".params."' ?") %c
    %and ->next command
  mrun = 1                {allow to continue after break}
  control latch =16_01    {remove reset condition}
  control latch =16_01    {start processor execution}
  ->monitor

com(init):
   initialise urisc
  ->nextcommand

com(dump):
  %if getfromtofile(from, to, file) %then dump area(from, to, file)
  ->nextcommand

com(go6809):
  %if params # "" %then printmon("parameters : `".params."' ?") %c
    %and ->next command
  
   run 6809 program
   ->nextcommand

com(help):
  print help
  ->nextcommand

com(infile):{nominate input file}
  %if params = "" %then %start
    prompt("file: ")
    readline(file)
  %finish %else file = params
  nominate infile(file)
  ->nextcommand

com(load6809):
  %if getfilead(file, ad) %then load file(file, ad)
  ->nextcommand

com(memtest):
  %IF getfromtoval(from,to,val) %START
    memory test(from,to,val,i)
  %FINISH
  ->nextcommand

com(object): {down-load motorola format object file}
  %if params = "" %then %start
    prompt("file: ")
    readline(file)
  %finish %else file = params
  load object file(file)
  load object file(downloadmonitor)    {always load monitor program}
  ->nextcommand

com(outfile):{nominate output file}
  nominate outfile(params)
  ->nextcommand

com(quit):
  %if params # "" %then printmon("parameters : `".params."' ?") %c
    %and ->next command
  %signal %event 0,1    {ctrl-y to terminate}
     
com(c registers):
  print registers
  ->nextcommand

com(reset6809):
  %if params # "" %then printmon("parameters : `".params."' ?") %c
    %and ->next command
  control latch =16_04   {reset bit}
  ->nextcommand

com(step):
    clock=0
    transmit
    clock=1
    transmit
    clock=0
    transmit
  ->nextcommand

com(transfer):
  %if getfromtofile(from, to, file) %then transfer to file(from, to, file)
  ->nextcommand

com(zero):
  %if params # "" %then printmon("parameters : `".params."' ?") %c
    %and ->next command
  zerostore
  ->nextcommand

com(c freerun):
 %if get bool("freerun =",flag) %then freerun=flag
 ->nextcommand

com(c clock):
 clock=flag %if get bool("clock =",flag)
 ->nextcommand

com(c go):
    go=flag %if get bool("go =",flag)
 ->nextcommand

com(c loadstore):
    loadstore=flag %if get bool("l/s =",flag)
  ->nextcommand

com(c mode):
    mode=flag %if get bool("mode =",flag)
  ->nextcommand

com(c dclk):
    dclk=flag %if get bool("dclk =",flag)
  ->nextcommand

com(c sdo):
    sdo=flag %if get bool("sdo =",flag)
  ->nextcommand

com(c reset):
    reset=flag %if get bool("reset =",flag)
  ->nextcommand

com(c load):
    load=flag %if get bool("load =",flag)
    ->next command

com(c transmit):
    transmit
    ->next command

com(c get):
%if   get registers %start
   print registers
%else
   printmon("failed")
%finish
  ->next command

com(c put):
printmon("failed") %unless    put registers
  ->next command

com(c set):
   %if get reg n val(reg,val) %then %start
       set register(reg,val)
     %finish
    ->next command   

com(c store):
printmon("failed") %unless   store registers
   ->next command

com(c write):
    %IF getadval(ad,val) %START
        %IF %NOT urisc write(ad,val) %THENC
            printmon("Write Unsuccessful")
    %FINISH
    ->next command

com(c read):
    val=0
    c=0
    %IF get adval(ad,n) %START
    %FOR i=ad,1,n %CYCLE
            phex(i)
            print string(" : ")
        %IF urisc read(i,val) %START
            phex(val)
            newline
        %ELSE
            printline("--------")
            c=c+1
            %exit %if c>10
        %FINISH
    %REPEAT
    %FINISH
    ->nextcommand

com(c soak):
    %If getadval(ad,val) %THEN soak test(ad,val)
    ->nextcommand

com(c download):
    %if params="" %then %start
        prompt("file: ")
        readline(file)
    %finish %else file=params
    urisc download(file)
   ->nextcommand

com(c execute):
    %if params="" %start
        val=1
    %else
      ->nextcommand %unless intparam(params,val)
    %finish
    urisc execute(val)
    ->nextcommand

com(c fullspeed):
  %if params#"" %start
        printmon("params?")
  %else
    full speed
  %finish
  ->nextcommand

{------------------------------------------------------------------------------}

\end{verbatim}
\begin{verbatim}

monitor:


  {check if character entered at APM keyboard}
  c = test symbol
  %if c >= 0 %then %start    {char typed at keyboard}
    %if c = ctrl c %then %signal %event 0,0
    %if keybufn > keybufl %then %start    {keybuf full}
      printsymbol(bel)
      printmon("input buffer full")
    %finish %else %start
      keybuf(keybufinp) = c
      keybufinp = (keybufinp+1) & keybufl
      keybufn = keybufn+1
    %finish
  %finish


  %if mrun = 0 %then haltm6809 %and ->nextcommand    {halt m6809}



  ->monitor

{------------------------------------------------------------------------------}

%routine printmon(%string(255) s)
  print string("** ".s." **
")
%end

{------------------------------------------------------------------------------}

%routine checkm6809
%integer i

  %on %event 0 %start
    %if event_sub = 2 %then %start
      printsymbol(bel)
      printmon("No M6809 board")
      %stop
    %finish %else %signal %event 0, event_sub, event_extra
  %finish

  i = memory(0)    {attempt to read from the M6809 board}

%end

{------------------------------------------------------------------------------}

%routine haltm6809

  mrun = 0
  control latch = 16_00

%end

{------------------------------------------------------------------------------}

%routine run 6809 program
! start running the most recently loaded 6809 object code program

  {save program's start address from reset vector}
  progstarth = memory(memtop-2)
  progstartl = memory(memtop-1)
  
  {put address of `go' sequence in reset vector}
  memory(memtop-2) = goadh
  memory(memtop-1) = goadl
  control latch =16_04            {reset}
  mrun = 1                        {set m6809 run flag}
  control latch =16_01            {remove reset condition}
  control latch =16_01            {start processor execution}
  
  
   {restore program's start address to reset vector}
   memory(memtop-2) = progstarth
   memory(memtop-1) = progstartl

   control latch =16_04            {reset}
   mrun = 1                        {set m6809 run flag}
   iflg = 0                        {initialise interrupt-type flag}
   control latch =16_01            {remove reset condition}
   control latch =16_01            {start processor execution}
%end

{---------------------------------------------------------------------}


{---------------------------------------------------------------------}

%routine initcomms

  vdutxs = 1                      {m6809 transmit register empty}
  keyrx = 10                      {dummy nl to start off with}
  keyrxs = 1                      {m6809 receive register full}
  keybufinp = 0                   {next empty hole in keybuf}
  keybufoutp = 0                  {next char to be sent to m6809}
  keybufn = 0                     {number of chars in keybuf}
  filtxs = 1                      {m6809 file transmit register empty}
  %if fileout # "" %then %start
    select output(2)
    reset output
    select output(0)
  %finish
  filrx = 10                      {dummy nl to start off with}
  filrxs = 1                      {m6809 file receive register full}
  %if filein # "" %then %start
    select input(2)
    reset input
    select input(0)
  %finish
  regtxs = 1                      {m6809 register transmit reg empty}
  fstxs = 1                       {frame store transmit register empty}
  cmtxs = 1                       {colour map transmit register empty}

%end

{----------------------------------------------------------------------}

%routine nominate infile(%string(80) file)

  %on %event 3 %start
    printmon("file `".filein."' does not exist")
    filein = ""
    select input(0)
    %return
  %finish

  %if filein # "" %then %start
    select input(2)
    close input
    select input(0)
    printmon("input file `".filein."' closed")
  %finish
  filein = file
  open input(2, filein)
  select input(0)

%end

{----------------------------------------------------------------------}

%routine nominate outfile(%string(80) file)

  %on %event 3 %start
    select output(0)
    printmon("cannot open file `".fileout."'")
    fileout = ""
    %return
  %finish

  %if fileout # "" %then %start
    select output(2)
    close output
    select output(0)
    printmon("output file '".fileout."' closed")
  %finish
  fileout = file
  %if fileout # "" %then open output(2, fileout)
  select output(0)

%end

{------------------------------------------------------------------------------}

%integer %fn get file char
%integer c

  %on %event 9 %start
    printmon("input file `".filein."' ended")
    select input(2)
    close input
    select input(0)
    filein = ""
    %result = ctrl z
  %finish

  select input(2)
  read symbol(c)
  select input(0)
  %result = c

%end

{----------------------------------------------------------------------}

%routine closefiles

  %if filein # "" %then %start
    select input(2)
    close input
    select input(0)
    printmon("input file `".filein."' closed")
    filein = ""
  %finish

  %if fileout # "" %then %start
    select output(2)
    close output
    select output(0)
    printmon("output file '".fileout."' closed")
    fileout = ""
  %finish

%end

{---------------------------------------------------------------------}

%string(80)%fn strip(%string(80) s)
%string(80) head, tail

  {strip leading spaces}
  s = tail %while s -> head.(" ").tail %and head = ""

  {remove multiple and trailing spaces}
  %cycle
    %exit %unless s -> head.(" ").tail
    %if tail = "" %then s = head %and %continue
    s = head." ".strip(tail)
    %exit
  %repeat

  s = head.",".tail %while s -> head.(" ,").tail %or s -> head.(", ").tail

  %result = s

%end

{----------------------------------------------------------------------}

%predicate intparam(%string(80) p, %integername ip)
{convert hex string `p' to integer value `ip'}
%string(80) q, r
%integer i, l, c

  l = length(p)
  %if l = 0 %then printmon("missing hex parameter") %and %false
  ip = 0
  %for i = 1, 1, l %cycle
    c = charno(p, i)    
    %if '0' <= c <= '9' %then ip = ip<<4+c-'0' %else %c
    %if 'a' <= c <= 'f' %then ip = ip<<4+c-'a'+10 %else %c
    %if 'A' <= c <= 'F' %then ip = ip<<4+c-'A'+10 %else %c
      printmon("invalid hex parameter: `".p."'") %and %false
  %repeat

  %true

%end

{----------------------------------------------------------------------}

%predicate getfromtofile(%integername from, to, %string(*)%name file)
%string(80) p, q, r, s

  %if params = "" %then %start
    prompt("from address: ")
    readline(p)
    prompt("to address: ")
    readline(r)
    prompt("to file: ")
    readline(file)
  %finish %else %start
    %if params -> p.(",").q %then %start
      %if p -> s.(" ").r %then p = s %and file = q %and ->check
    %finish %else %start
      %unless params -> p.(" ").q %then %c
        printmon("invalid parameters: `".params."'") %and %false
    %finish
    {optional file name}
    %unless q -> r.(",").file %then %start
      %unless q -> r.(" ").file %then r = q %and file = ""
    %finish
  %finish

check:
  %false %unless intparam(p, from)
  %false %unless intparam(r, to)

  %true

%end

{--------------------------------------------------------------------}

%predicate getadval(%integername ad, val)
%string(80) p, q

  %if params = "" %then %start
    prompt("address: ")
    readline(p)
    prompt("value: ")
    readline(q)
  %finish %else %start
    %unless params -> p.(",").q %then %start
      %unless params -> p.(" ").q %then %c
        printmon("invalid parameters: `".params."'") %and %false
    %finish
  %finish
  %false %unless intparam(p, ad)
  %false %unless intparam(q, val)

  %true

%end

{------------------------------------------------------------------------------}

%predicate getfromtoval(%integername from, to, val)
%string(80) p, q, r, s, t

  %if params = "" %then %start
    prompt("from address: ")
    readline(p)
    prompt("to address: ")
    readline(r)
    prompt("value: ")
    readline(s)
  %finish %else %start
    %if params -> p.(",").q %then %start
      %if p -> t.(" ").r %then p = t %and s = q %and ->check
    %finish %else %start
      %unless params -> p.(" ").q %then %c
        printmon("invalid parameters: `".params."'") %and %false
    %finish
    %unless q -> r.(",").s %then %start
      %unless q -> r.(" ").s %then %c
        printmon("invalid parameters: `".params."'") %and %false
    %finish
  %finish

check:
  %false %unless intparam(p, from)
  %false %unless intparam(r, to)
  %false %unless intparam(s, val)

  %true

%end

{------------------------------------------------------------------------------}

%predicate getfilead(%string(*)%name file, %integername ad)
%string(80) p

  %if params = "" %then %start
    prompt("file: ")
    readline(file)
    prompt("address: ")
    readline(p)
  %finish %else %start
    %unless params -> file.(",").p %then %start
      %unless params -> file.(" ").p %then %c
        printmon("invalid parameters: `".params."'") %and %false
    %finish
  %finish
  %false %unless intparam(p, ad)
  %true

%end

{------------------------------------------------------------------------------}

%routine print help

  printline("area  from,to,value       -  set memory(from..to) all to `value'" )
  printline("byte  addr,value          -  set memory(addr) to byte `value'" )
  printline("continue                  -  allow program to continue")
  printline("dclk                      -  set dclk value")
  printline("dump  from,to [,file]     -  dump memory(from..to) to VDU or file")
  printline("run                       -  reset and start program")
  printline("help                      -  display this help information")
  printline("infile  file              -  nominate `file' as input file")
  printline("ld  file,from             -  load file to memory starting at `from'")
  printline("memtest                   -  perform read/write memory test")
  printline("object  file              -  download Motorola object format file" )
  printline("outfile  file             -  nominate `file' as output file")
  printline("quit                      -  quit from monitor" )
  printline("registers                 -  print register values on VDU")
  printline("rst                       -  halt and reset M6809" )
  printline("step                      -  step the URISC clock" )
  printline("transfer  from,to [,file] -  transfer memory(from..to) to VDU or file" )
  printline("zero                      -  clear whole 6809 memory to zero" )
  printline("freerun                   -  set freerun value")
  printline("clock                     -  set clock value")
  printline("go                        -  set go value")
  printline("l/s                       -  modify l/s signal")
  printline("sdo                       -  modify serial data out")
  printline("load                      -  modify load signal")
  printline("mode                      -  modify mode value")
  printline("reset                     -  modify reset signal")
  printline("tx                        -  transmit signals to URISC")
  printline("get                       -  get URISC registers")
  printline("put                       -  put back URISC registers")
  printline("set reg,value             -  set register to value")
  printline("init                      -  initialise URISC")
  printline("execute    <n>            -  execute instructions")
  printline("spin                      -  free run instr. execution")
  newline
  printline("command names may be abbreviated; parameters are hexadecimal")
  printline("ctrl-c breaks; ctrl-y terminates; ctrl-z ends input")

%end

{------------------------------------------------------------------------------}

%routine dump area(%integer from, to, %string(80) file)
%integer ad, m, i, same, ditto

  %if file # "" %then open output(2, file) %and select output(2)

  newline

  from = from&16_fffffff0
  %if from < 0 %then from = 0
  to = (to+16)&16_fffffff0
  %if to > memtop %then to = memtop
  to = to-16
  %if to < from %then to = from

  %for ad = from, 16, to %cycle

    {test for same as previous line and next line}
    %if ad > from %and ad < to %then %start
      same = 1
      %for i = 0, 1, 15 %cycle
        %if memory(ad+i) # memory(ad+i-16) %or %c
            memory(ad+i) # memory(ad+i+16) %then same = 0 %and %exit
      %repeat
    %finish %else same = 0

    %if same = 0 %then %start
      print string(hexword(ad)." : ")
      %for i = 0, 1, 15 %cycle
        print string(hexbyte(memory(ad+i))." ")
      %repeat
      spaces(5)
      %for i = 0, 1, 15 %cycle
        m = memory(ad+i)
        %if m < 32 %or m >= 127 %then m = '.'
        print symbol(m)
      %repeat
      newline
      ditto = 0
    %finish %else %start
      %if ditto = 0 %then printline("       -- ditto --") %and ditto = 1
    %finish

  %repeat

  %if file # "" %then close output
  select output(0)

%end

{------------------------------------------------------------------------------}

%routine load object file(%string(80) file)
{download a motorola format object file}
%string(80) head, tail
%integer ad, i, b, c, l, t, checksum
%integerfnspec getbyte   
%constant %string(31) %array err(1:6) = %c
  "invalid hex digit",
  "invalid start code",
  "unknown record type",
  "invalid record length",
  "invalid address",
  "checksum error"

  %on %event 9, 15 %start
    %if event_event = 9 %then printmon("incomplete object file") %c
      %else printmon("corrupt object file (".err(event_sub).")")
    ->endin
  %finish

  %unless file -> head.(".").tail %then file = file.".obj"
  %unless exists(file) %then printmon( %c
    "file `".file."' does not exist") %and %return

  open input(2, file)
  select input(2)

  %cycle    {loop for all records in object file or until an error occurs}

    read symbol(c) %until c > ' '    {skip until significant char}
    %if c # 'S' %then %signal %event 15,2

    read symbol(t)    {record type}

    checksum = 0    {start checksum after type byte}
    l = getbyte

    %if t = '0' %or t = '1' %or t = '9' %then ad = getbyte<<8!getbyte %c
      %and l = l-3 %else %c
    %if t = '2' %or t = '8' %then ad = getbyte<<16!getbyte<<8!getbyte %c
      %and l = l-4 %else %c
    %if t = '3' %or t = '7' %then ad = getbyte<<24!getbyte<<16!getbyte<<8 %c
      !getbyte %and l = l-5 %else %signal %event 15,3    {unknown record type}

    %if l < 0 %then %signal %event 15,4    {funny length byte}
    %if ad < 0 %then %signal %event 15,5    {funny load address}

    %for i = ad, 1, ad+l-1 %cycle
      b = getbyte
      memory(i) = b %unless i >= memtop %or t = '0'    {Fred's dummy header}
    %repeat

    c = getbyte    {checksum byte}
    %if checksum&255 # 16_ff %then %signal %event 15,6    {checksum error}

  %repeat %until t >= '7'    {end record}

endin:
  close input
  select input(0)    
  %return

{------------------------------------------------------------------------------}

%integerfn getbyte
%integer b, c

  read symbol(c)
  %if '0' <= c <= '9' %then b = c-'0' %else %c
  %if 'A' <= c <= 'F' %then b = c-'A'+10 %else %signal %event 15,1
  read symbol(c)
  %if '0' <= c <= '9' %then b = b<<4+c-'0' %else %c
  %if 'A' <= c <= 'F' %then b = b<<4+c-'A'+10 %else %signal %event 15,1
  checksum = checksum + b
  %result = b

%end

%end

{------------------------------------------------------------------------------}

%routine load file(%string(80) file, %integer from)
%integer ad, c

  %on %event 9 %start; ->endin; %finish

  %unless exists(file) %then printmon( %c
    "file `".file."' does not exist") %and %return

  open input(2, file)
  select input(2)

  ad = from
  %cycle
    read symbol(c)
    memory(ad) = c
    ad = ad+1
  %repeat    {loop until `end of input'}
      
endin:
  close input
  select input(0)

%end

{------------------------------------------------------------------------------}

%routine transfer to file(%integer from, to, %string(80) file)
%constinteger nl = 10
%integer ad, c

  %if from > to %then printmon("memory area inside out") %and %return
  %if from > memtop %then printmon("memory area out of range") %c
    %and %return
  %if to >= memtop %then to = memtop-1
  
  %if file # "" %then open output(2, file) %and select output(2)
  
  ad = from
  %cycle
    c = memory(ad)
    {avoid printing non-printing chars to VDU}
    %if file = "" %and c < ' ' %and c # nl %then c = '.'
    print symbol(c)
    ad = ad+1
  %repeat %until ad > to

  %if file # "" %then close output %else newline
  select output(0)

%end

{------------------------------------------------------------------------------}

%routine store byte(%integer ad, val)

  %unless 0 <= ad < memtop %then printmon("address out of range") %c
    %and %return
  %if periphbot <= ad < periphtop %then printmon( %c
    "address in peripherals area") %and %return
  %if val > 255 %then printmon("value too large") %and %return
  memory(ad) = val

%end

{------------------------------------------------------------------------------}

%routine store area(%integer from, to, val)
%integer i

  %if from > to %then printmon("area inside out") %and %return

  %unless 0 <= from < memtop %and 0 <= to < memtop %then %c
    printmon("address out of range") %and %return
  %if periphbot <= from < periphtop %or periphbot <= to < periphtop %or %c
    (from < periphbot %and to >= periphtop) %then %c
    printmon("address in peripherals area") %and %return

  %if val > 255 %then printmon("value too large: ".itos(val,0)) %c
    %and %return

  %for i = from, 1, to %cycle
    memory(i) = val
  %repeat

%end

{------------------------------------------------------------------------------}


%string(2)%fn hexbyte(%integer i)
%conststring(1)%array hex(0:15) = "0","1","2","3","4","5","6","7",
  "8","9","A","B","C","D","E","F"
  %result = hex((i>>4)&15).hex(i&15)
%end

{------------------------------------------------------------------------------}

%string(4)%fn hexword(%integer i)
  %result = hexbyte(i>>8).hexbyte(i&255)
%end

{------------------------------------------------------------------------------}


{------------------------------------------------------------------------------}

%routine zerostore
%integer i

  %for i = 0, 1, memtop-1 %cycle
    memory(i) = 0
  %repeat

%end
\end{verbatim}
\begin{verbatim}

{*****************************************************************************}


!                 Ultimate RISC code Starts Here


{*****************************************************************************}

! First comes PIA control

%routine extract info(%byte PortA)
   state=PortA & 31
   halt=(PortA & 32) >> 5
   sdi=(PortA & 64) >> 6
%END

%ROUTINE form data(%bytename pa,pb)
! converts all the lines into2 bytes of data

   %signalevent 11,4 %if dclk=1 %and sdo=1 %and mode=1

   pa=sdo<<7
   pb=(reset<<7)+(dclk<<6)+(mode<<5)+(loadstore<<4)+   %c
      (load<<3)+(go<<2)+(clock<<1)+freerun
%END

%routine pause
%integer i
 %for i=1,1,500 %cycle
 %repeat
%end

%ROUTINE wait for PIA

! waits for the PIA
! but times out eventually
! and then raises event 11,3
%INTEGER i

   i=20000
   %while addr#0 %cycle
        i=i-1
        %signalevent 11,3 %if i=0
    %repeat
%END


%ROUTINE write to PIA (%BYTE a,d)
!waits till last write has been completed
! then writes data d to pia register a

   wait for PIA
   data=d
   addr=a!128
%END

%ROUTINE transmit
! forms the data bytes and transmits to the host
! precondition: data registers have been selected

%BYTE pa,pb

   form data(pa,pb)
       write to PIA(dataA,pa)
   write to PIA(dataB,pb)
%END

%ROUTINE set up PIA
! initialisation routine to set up the PIA
! correctly for the host i/f

   write to pia(ctrla,2_00000000)
   write to pia(dataA,2_10000000)
   write to pia(ctrlA,2_00000100)
   write to pia(ctrlB,2_00000000)
   write to pia(dataB,2_11111111)
   write to pia(ctrlB,2_00000100)
   val1=0
   val2=0
   pause reg=0
   times reg=0
%END

%ROUTINE clock cpu
! clocks the EPLDs of the URISC

   clock=0
   freerun=1
   transmit
   clock=1
   transmit
   clock=0
   transmit
%END

%ROUTINE clock shadow
! clocks the dclk inputs of all the SSR registers
! assumes the user has
! suspended clocking of the control unit

   dclk=0
   transmit
   dclk=1
   transmit
   dclk=0
   transmit
%END

%PREDICATE shadow clock loop(%byte n)
! will only work for extended 6809 PIA control s/w
! -instructs 6809 to clock the shadow register the given number of times
%integer timeout
%byte pa

    pause reg=0
    dclk=0
    form data(pa,val1)
    write to pia(dataA,pa)
    dclk=1
    form data(pa,val2)
    dclk=0
    wait for pia
    timeout=16_2000
!   printline("repeating ".itos(n,0)." times")
    %if n#0 %start
           times reg=n
           %while times reg#0 %and timeout#0 %cycle
               timeout=timeout-1
           %repeat
           %if timeout=0 %start
               printmon("Couldn't get 6809 to clock shadow in a loop")
               printmon("Is the correct PIA program installed?")
               times reg=0
        %false
    %finish
    %finish
        %true
%END

%PREDICATE clock loop(%byte n)
%integer timeout
%byte pa
    pause reg=0
    clock=0
    form data(pa,val1)
    write to pia(dataA,pa)
    clock=1
    form data(pa,val2)
    clock=0
    wait for PIA
    %if n#0 %start
        timeout=2000
        times reg=n
        %while times reg#0 %and timeout#0 %cycle
            timeout=timeout-1
        %repeat
        %if timeout=0 %start
            printmon("timeout on 6809 PIA- is the program active?")
            %false
        %finish
    %finish
    %true
%end
{********************************************************************}

{ The Control unit states

%ROUTINE read state names
!reads in all the state names from the named file
! assumes the ordering of names is correct
! and there are 31 to read in

%integer i
   open input(2,state file)
   select input(2)
   %for i=0,1,31 %cycle
      read line(state names(i))
   %repeat
   close input
   select input(0)
%END

%ROUTINE get state
! waits until last operation is over
! before reading and decoding the information from the URISC
%BYTE temp

   wait for PIA
   temp=pia copy(dataA)
   extract info(temp)
%END

{**********************************************************************}
{                       The Registers                                  }
{**********************************************************************}

%ROUTINE update banks(%STRING(stringsize) reg,%BYTE s)
%INTEGER i,found

    found=0
    %for i=1,1,register banks %cycle
        %if bank names(i)=reg %start
            bank sizes(i)=s %if s>bank sizes(i)
            found=1
            %finish
        %repeat
    %if found=0 %start
        register banks=register banks+1
        bank sizes(register banks)=s
        bank names(register banks)=reg
    %finish
%END

%INTEGERFN register size(%STRING(*)%NAME r)
! returns the no of byte size registers making up a named bank
%INTEGER i
    %for i=1,1,register banks %cycle
        %result=bank sizes(i)  %if bank names(i)=r
    %repeat
    %result=-1
%end
        

%ROUTINE read register names
! reads in the register file
! format
!  # regs
!  register1       subscript1
!            ...
!  registerj       subscriptj
!
!  register is the name of the register e.g Instruction
!  subscript gives the position of that byte within the named register
!  e.g 0 = lsbyte , 3 =ms byte of 32 bit register

%INTEGER i

   register banks=0
   open input(2,register file)
   select input(2)
   read(registers)
   %for i=1,1,registers %cycle
      read(register names(i))
      read(register subscripts(i))
      update banks(register names(i),register subscripts(i))
   %repeat
   close input
   select input(0)

%END

%INTEGERFN find register(%string(stringsize) r, %integer s);
! finds the register with given name and subscript
%INTEGER i

   %FOR i=1,1,registers %CYCLE
      %IF register names(i)=r %AND register subscripts(i)=s %THENC
            %RESULT=i
   %REPEAT
   %RESULT=-1
%END

%INTEGERFN register value(%string(stringsize) r)
%INTEGER i,reg,v,b
   v=0
   b=register size(r)
   %if b>=0 %start
      %for i=0,1,b   %CYCLE
          reg=find register(r,i)
          v=v+register values(reg)<<(8*i)
       %REPEAT
      %finish
   %RESULT=v
%END

%ROUTINE print registers
%BYTE i
%BYTE size
%INTEGER value=0

    newline
    write(register banks,0)
    print string(" registers built from ")
    write(registers,0)
    print line(" SSRs")
    newline
    %if registers>0 %start
      %for i = 1,1,register banks %cycle
        size=bank sizes(i)+1
        value=register value(bank names(i))
        print string(bank names(i)." : ")
        %if size=1 %then phex2(value) %elsec
        %if size=2 %then phex4(value) %elsec
                phex(value)
        newline
    %repeat
    %finish
%END


%ROUTINE write back to registers(%STRING(stringsize)r,%integer v,l)
! r=reg name
! v=value
! l = length in bytes
%INTEGER i,reg

    i=0
    %CYCLE
        reg=find register(r,i)
        %IF reg=0 %THEN printmon("Cant find the register!") %ELSEC
            register values(reg)=v&255
        v=v>>8
        i=i+1
    %REPEAT %UNTIL i>l
    %IF v>0 %THEN printmon(" Value Too Large- extra ignored")
%END

%ROUTINE set register(%STRING(*)%NAME  r,%INTEGER v)

%INTEGER b,i
    b=register size(r)
    %IF b=-1 %THEN printmon("No such register") %ELSEC
            write back to registers(r,v,b)
%END


%PREDICATE clear shadow
! clears all the shadow registers by shifting a large number
! of zeros in.
%integer i

    SDO=0
    MODE=0
!    %for i=1,1,registers*8 %cycle
!        clock shadow
!    %repeat
    %false %unless shadow clock loop(8*registers)

    get state
    %if sdi#0 %start 
        printmon("`0' not passed through SSR chain while clearing shadows")
        %false
    %finish
    %true
%end


%PREDICATE copy registers to shadow
! instructs the ssr chain to copy its Y reg. to its shadow
! and propagates the command along all the registers.
! raises event 11,0 

%INTEGER i

   %false %unless clear shadow
   sdo=0
   mode=1
   clock shadow
   get state
   mode=0
!   %SIGNALEVENT 11,0 %IF sdi#0
   %IF sdi#0 %start
        printmon("'0' not passed through SSR chain while copying registers")
        %false
    %finish

    transmit
        %true
%END

%PREDICATE get registers
%INTEGER i,j,temp,timeout
! copies all the registers to the array

! signals event 11,1 if a 1 was not propagated through the chain
! -the definitive test for the presence of the computer-

%false %unless   copy registers to shadow 
!   temp=1
!  %for i=1,1,registers %cycle
!     register values(i)=0
!     %for j=0,1,7 %cycle
!        sdo=temp
!        temp=sdi
!        get state
!        register values(i)=register values(i)<<1 + sdi
!        sdo=sdi
!        clock shadow
!     %repeat
!  %repeat
!  sdo=temp
!  temp=sdi
! clock shadow
   %if shadow clock loop(0) %start
     %finish
   timeout=timelimit
   pia regs=registers
   pia dir=129
   %while pia dir#0 %and timeout#0 %cycle
    timeout=timeout-1
    %repeat
   %if pia dir#0 %start
        printmon("PIA timeout on get")
        %false
    %finish

   get state
!  %signalevent 11,1 %if temp#1
!   %if temp#1 %start
!        printmon("`1' not passed through SSR chain during register put")
!        %false
!        %finish
  %true
%END

%PREDICATE put registers
! puts the stored registers back into the SSR chain
! checks to see if sdi is equal to the highest bit
! in the last register & raises 11,0 or 11,1 if not
%INTEGER i,j,timeout

   %FOR i=1,1,registers %CYCLE
      %FOR j=7,-1,0 %CYCLE
         sdo=(register values(i)>>j)&1
         clock shadow
      %REPEAT
    %REPEAT
   i=(register values(1)>>7) & 1
   get state
!   %signalevent 11,i %IF sdi#i
   %if sdi#i %start
        printmon("`".itos(i,0).%c
 "' not passed through SSR chain while putting registers")
         %false
    %else
     %true
    %finish
!  %if shadow clock loop(0) %start
!           %finish
!  timeout=timelimit
!  pia regs=registers
!  pia dir=128
!  %while pia dir#0 %and timeout#0 %cycle
!     timeout=timeout-1
!  %repeat
!  %if timeout=0 %start
!       printmon("PIA timeout on put")
!       %false
!   %else
!       %true
!   %finish
%END

%PREDICATE store registers
! puts back the registers & then stores them using the control
! EPLD - will only work if in the halted or waiting states
    get state
    %if state=halted state %START
        load=1
        go=1
        clock cpu
        get state
    %FINISH
    %IF state#waiting state %START
        printmon("Not in the right state")
        %false
    %ELSE
        load=1
        go=1
        %false %unless put registers
        mode=1
        load=0
        clock cpu
        load=1
        mode=0
        transmit
        %true
    %FINISH
%END

%PREDICATE  urisc write(%INTEGER addr,data)
! tries to write to the urisc memory
! returns false if it couldn't (control unit problems)

    set register(MAR,addr)
    set register(MDR,data)
    %false %unless store registers
    get state
    %IF state=loaded state %START
        loadstore=0
!       %WHILE {state#halted state %AND} state#waiting state %C
!                 %CYCLE
           
!           clock cpu
    %false %unless clock loop(5)
            get state
!        %REPEAT
        %if state=waiting state %then %true %else %false
    %ELSE
        %FALSE
    %FINISH
%END

%PREDICATE urisc read(%INTEGER addr,%INTEGERNAME data)
! tries to read location addr and put the word therein into
! the argument 'data'
! returns false if ic could not do this- control unit problems
! or because that address is write only...

    set register(MAR,addr)
    %false %unless store registers
    data=0
    get state
    %if state=loaded state %START
        loadstore=1
!     load=0
!       %false %unless clock loop(4)
        %WHILE state#halted state %AND state#waiting state %c
            %CYCLE
           clock cpu
            get state
         %REPEAT
        %IF state=waiting state %START
            %false %unless get registers
            data=register value(MDR)
            %TRUE
        %ELSE
            data=0
            %FALSE
        %FINISH
    %ELSE
            data=0
            %FALSE
    %FINISH
%END

%PREDICATE step to start
! tries to step the urisc to the start state,
! returning true if it does so
%integer count=0

    get state
    %while state#waiting state %and state#halted  state %andc
        count<32 %cycle
            clock cpu
            get state
    %repeat
    %true %if state=waiting state %or state=halted state
    %false
%END
        
 
%ROUTINE memory test(%INTEGER from,to,value, %INTEGERNAME fail)
! tests the region of memory specified
! if value=from then increments the value to the address every time
! - a check for folding of memory
%integer i,v,data

    fail=0
    v=value
    print string("Memory test from ")
    phex(from)
    print string(" to ")
    phex(to)
    print string(" with ")
    %if value=from %thenc
        print string(" address ") %elsec
        phex(value)
    newline
    %FOR i=from,1,to %CYCLE
        v=i %IF from=value
        %IF %NOT urisc write(i,v) %START
            phex(i)
            print line(" : couldn't write")
            fail=fail+1
            %Unless step to start %start
                print line("Couldnt get to a starting state-terminating test")
                %return
            %finish
        %FINISH
    %REPEAT

    ! now read back & verify

    %FOR i=from,1,to %CYCLE
        %IF urisc read(i,data) %START
            v=i %IF from=value
            %IF v#data %START
                phex(i)
                print string(" : failed - ")
                phex(data)
                newline
                fail=fail+1
            %FINISH
        %ELSE
            phex(i)
            print line(" : write only")
            fail=fail+1
        %FINISH
    %REPEAT

    %IF fail=0 %START
            newline
            print line("Memory test successful")
    %ELSE
            newline
            print string("Failed during memory test ")
            write(fail,0)
            print string(" time")
            print symbol('s') %IF fail#1
            newline
    %FINISH
%end

%ROUTINE soak test(%integer reps,failures)
! tests memory for either the specified number of repetitions
!  (0=dont stop)
! or until the given number of failures is exceeded-
!  0 = don't stop
! testing process is:
!  1. checkerboard of bits 5555555
!  2. (1) inverted -aaaaaaaa
!  3. (addr)<-addr   (folding test)

! checks for failures are only made between each sub test

%INTEGER soaks,failure count,subtest,subfails,pattern=0

    soaks=0
    failure count=0
    
    %WHILE (soaks<reps %OR reps=0) %ANDC
           (failure count<failures %OR failures=0) %CYCLE

        soaks=soaks+1
        subtest=1
        subfails=0
        %WHILE (subtest<=3) %AND (failure count<failures %ORC
                                 failures=0) %CYCLE
            pattern=16_55555555 %IF subtest=1
            pattern=16_AAAAAAAA %IF subtest=2
            pattern=16_20000000 %IF subtest=3
            memory test(16_2000,16_3fff,pattern,subfails)
            failure count=failure count+subfails
            subtest=subtest+1
            printline("Repetition :".itos(soaks,0)."  total no. of failures:".%c
                       itos(failure count,0))
        %REPEAT
    %REPEAT
%END


%ROUTINE urisc download(%STRING(80) file)
! downloads the given file into memory
! the format is that agreed with RBE
! 1. all numbers in Hex
! 2. any number of code sequences
! 3. each sequence consists of a start address, a length
!    and then the code words
! 4. comments, preceeded by an exclamation mark will be displayed.
!    they can appear anywhere in the file
! 5. each number should be on a separate line

%integer ad,length=0,count=0
%string(80) line,comment,addr

%on %event 9 %start
    ->endin
    %finish

    %unless exists(file) %thenc
        printmon("File does not exist") %and %return
    open input(2,file)
    select input(2)

! repeat till the end of the file
    %CYCLE
        {get the start address}
        ad=-1
        length=-1
        count=0
        %while ad=-1 %cycle
            read line(line)
            line=strip(line)
            %if line#"" %start
                %if line->("!").comment %start
                    print line(comment)
                %else
                %if line->("@").addr %start
                    addr=strip(addr)
                    %unless intparam(addr,print addr) %start
                        printmon("Don't understand @ address")
                        ->endin
                    %finish
                %else
                   ->endin %unless intparam(line,ad)
                %finish
                %finish
            %finish
        %repeat
        %if ad>32767 %thenc
            printmon("address out of range") %and ->endin

        {now get the length}
        %while length=-1 %cycle
            read line(line)
            line=strip(line)
            %if line#"" %start
                %if line->("!").comment %start
                    print line(comment)
                %else
                    ->endin %unless intparam(line,length)
                %finish
            %finish
        %repeat
    
        {now read in the code}
        %if length=0 %thenc
            printmon("Code seqence of length zero!") %and ->endin

        print string("Loading code sequence from ")
        phex(ad)
        print string(" to ")
        phex(ad+length-1)
        newline

        %while count<length %cycle
            read line(line)
            line=strip(line)
            %if line#"" %start
                %if line->("!").comment %start
                    print line(comment)
                %else
                    %if intparam(line,value) %start
                        ->endin %unless urisc write(ad+count,value)
                        count=count+1
                    %else
                        ->endin
                    %finish
                %finish
            %finish
        %repeat
    %REPEAT

endin:
    %if ad#-1 %and count<length %thenc
        printmon("premature end of download")
    close input
    select input(0)

%END

%PREDICATE print halt
! checks if halted with pc pointing at a RAM address with an 
! instruction of 16_00000001
! and if so prints all the contents of print addrr, then
! continues

%integer instr,char,i,chars

    %false %unless get registers
    %false %unless urisc read(register value(PC),instr)
    %false %unless instr=1
    %false %unless urisc read(print addr,chars)
    %for i=24,-8,0 %cycle
        char=(chars>>i)&255
        print symbol(char) %if char#0
    %repeat
    %true
%end

%ROUTINE urisc execute(%integer n)
! executes n instructions
! or till the halt state is reached
! if n=0 then run till halted
%integer count=0,key=-1

    get state
    %if state=halted state %start
        load=1
        go=1
        clock cpu
        get state
    %finish
    %if state#waiting state %start
        printmon("Not in the right state yet")
        %return
    %finish
    load=1
    %while (count<n %or n=0) %and state=waiting state %c
        %cycle
        count=count+1
        go=0
!       clock cpu
!       %cycle
!           clock cpu
            %if clock loop(9)  %start
                              %finish
            get state
        %if state#halted state %start
            go=1
            clock cpu
            get state
        %finish
!       %repeat %until state=halted state %or state=waiting state
        %if state=halted state %start
            %exit %unless print halt
        %finish
             key=test symbol
        %if key>=0 %start
            %if key=ctrl c %then %signalevent 0,0
        %finish
    %repeat
    newline
    print line("Instructions executed:".itos(count,0))
%END

%routine full speed
! a version of execute which instructs the URISC to execute at full speed;
! until halted or ^C is pressed.
! still supports Output

%integer stopit=0,key

printmon("Commencing Instruction Execution")

    %cycle
        go=0
        transmit
        freerun=0
        transmit
        %cycle
            get state
            update state
            key=test symbol
            %if key=3 %start
                go=1
                transmit
                stopit=1
                %cycle
                    get state
                %repeat %until state=halted state %or state=waiting state
            %finish
        %repeat %until state=halted state %or state=waiting state
    freerun=1
    transmit
    %repeat %until stopit=1  %or %not print halt
%end
        

{***********************************************************************}

!   User Interface stuff (HCI - what's that?)

{***********************************************************************}

%PREDICATE get bool(%STRING(40) p,%BYTENAME bool)
! if the parameters were empty then puts up the prompt & reads in the reply
! succeeds iff the reply = [0..1]

%STRING(80) s
%INTEGER ok

    %if params="" %start
        prompt(p)
        readline(s)
    %finish  %else s=params
    %if intparam(s,ok) %and((ok=0)%or(ok=1)) %thenc
          bool=ok %and  %true  %elsec
          printmon("0 or 1 only") %and %false
%END

%PREDICATE get reg n val(%STRING(*)%NAME reg,%INTEGERNAME ad)
! gets the register 'n' the value from the 'user'

%STRING(80) p

    %IF params="" %START
        prompt("register :")
        readline(reg)
        prompt("value :")
        readline(p)
    %ELSE
        %UNLESS params -> reg.(",").p %START
            %UNLESS params -> reg.(" ").p %THENC
                printmon("Invalid Parameters: `".params."'") %ANDFALSE
        %FINISH
    %FINISH
    %FALSE %UNLESS intparam(p,ad)
    %TRUE
%END



!A few routines to manage the wyse screen on the simulator


%ROUTINE reverse text
  print string(escape."[1m")
%END

%ROUTINE normal
   print string(escape."[m")
%END

%ROUTINE move to(%BYTE y,x)
  print string(escape."[")
  write(y,0)
  print symbol(';')
  write(x,0)
  print symbol('H')
%END

%ROUTINE wyse scroll region(%BYTE t,b)
  print string(escape."[")
  write(t,0)
  print symbol(';')
  write(b,0)
  print symbol('r')
%END


%ROUTINE set(%short m,%byte value)
  print string(escape."[")
  %IF m>0 %START

     write(m,0)
   %ELSE
     print symbol('?')
     write(-m,0)
   %FINISH
     %IF value#0 %THEN print symbol('h') %ELSE print symbol('l')
 %END


%ROUTINE print bit(%string(255)bit,%byte flag,x,y)
! prints the string at the given xy location, setting reverse
! if the flag is true

    move to(y,x)
    %if flag#0 %then reverse text %else normal
    print string(bit)
%END

%ROUTINE say(%byte x,y,%string(255) text)
! prints the text at the given place

    move to(y,x)
    print string(text)
%END

%routine frame
!draws the frame of the status box that doesn't need regular updating 
print string(escape."7")
    say(30,1,"---The Ultimate RISC---")
    say(1,2,"State #")
    say(40,2,"State Name :")
    move to(5,1)
    reverse text
    print symbol(32) %FOR i=1,1,80
    normal
    wyse scroll region(6,24)
    print string(escape."8")
%END

%routine update state
! update the state display of the screen

    print string(escape."7")
    say(8,2,itos(state,3))
    say(52,2,state names(state)."             ")
    print string(escape."8")
%end


%ROUTINE update
! update screen status display
%INTEGER i
    get state
    update state

    ! record postion/modes etc
    print string(escape."7")
    print bit("SDI",sdi,1,3)
    print bit("HALT",halt,15,3)
    print bit("SDO",sdo,30,3)
    print bit("Dclk",dclk,45,3)
    print bit("Mode",mode,60,3)
    print bit("Freerun",freerun,1,4)
    print bit("Clock",clock,15,4)
    print bit("Go",go,30,4)
    print bit("l/s",loadstore,45,4)
    print bit("Reset",reset,60,4)
    print bit("Load",load,75,4)
    normal 
    wyse scroll region(6,24)
    ! return to where i was before
    print string(escape."8")
%END

%ROUTINE initialise URISC
! the routine to set up the pia & then pass a reset to the urisc
!  (pc<-0)
! will then read in the register and state stuff off file
! and do the screen

    set up PIA
    sdo=0
    clock=0
    freerun=1
    go=1
    load=1
    loadstore=1
    mode=0
    dclk=0
    reset=0
    transmit
    reset=1
    transmit
    
    read register names
    read state names

    print string(escape."[2J")
    frame
    update
    move to(6,1)
    %if state=31 %start
        printmon("Is the Ultimate RISC installed?")
    %else
        print registers %if get registers
    %finish
    
%END

%endofprogram
\end{verbatim}
