\chapter{Memory}
\section{Overview}
In most computers  memory is used for storage of programs and data.
In many microcomputers peripheral elements are  also interfaced to the memory busses,
so that reading or writing certain locations actually controls the peripherals.
This obviates the need for special I/O instructions and control lines.
The technique is called {\em memory mapping}. 
It is useful where the composition  of systems varies widely, as different peripherals can be easily memory mapped. 
Example peripherals include a display controller, a disc controller, Ethernet communications link, keyboard decoder...etc. 

Since the only instruction available upon the Ultimate RISC is a memory to memory move, {\em all} functional elements of the system must be memory mapped, including the computer's main registers (table~\ref{table:memory}).
While this makes the control unit simple, part of the burden of complexity is passed to memory, remaining to extract a price on performance.


\begin{table}
\centering
\begin{tabular}{||lrll||}
\hline
\hline

Unit & Address	&	Name & 	 Function\\
\hline
Execution Unit  & & &\\
Write Only &  & &\\
& 0	 & PC		& program counter \\
& 1	  & SKIP	& skip register\\
& 2	 & X		& X index register \\
& 3	 & Y		& Y index register\\
& 4--7 & & duplicates of the above\\
 \hline
ALU: & &  &\\
Writing & & &\\
& 8--F	 & Carry & $Carry \leftarrow data_{0}$\\
& 10	 & CLR	& $Acc \leftarrow 0$\\
& 11	 & SUBR	& $Acc \leftarrow data - Acc$\\
& 12	 & SUB	& $Acc \leftarrow Acc - data$\\
& 13	 & ADD 	& $Acc \leftarrow Acc + data$\\
& 14  	 & XOR  & $Acc \leftarrow Acc \oplus data$\\
& 15	 & OR 	& $Acc \leftarrow Acc \vee data$\\
& 16	 & AND	& $Acc \leftarrow Acc \wedge data$\\
& 17	 & SET 	& $Acc \leftarrow$ \$FFFFFFFF\\
& 18--31 &  	& 10--17 followed by  a  one bit rotate right\\
\hline
ALU: & & &\\
Reading & & &\\
& 8--F  & Acc & Accumulator\\
& 10 & Z & Zero Flag\\
& 11 & N & Negative Flag\\
& 12 & V & Overflow Flag (not supported)\\
& 13 & C & Carry Flag\\
& 14--31 & & 10--13 repeated\\
\hline
Memory: & & &\\
bidirectional & & &\\
& 2000  & RAM		& Random Access Memory\\
& 3FFF &   &\\
\hline
\hline
\end{tabular}

\caption{Memory Map}
\label{table:memory}
\end{table}


\section{Address Decoding}

The memory of the computer needs to be informed when a memory access is required. It must be told the direction of the access, the address to be
used, and, on a write operation, the data which it must write.

During a read cycle it will return either the data stored at that location, or a signal indicating the address was invalid.

Some write operations  trigger one or more signals to different units of the computer, to update internal registers.

A means is needed  of interpreting the address and control signals to generate the required outputs. This is called  {\em address decoding}.

The first decision to be made when designing the address decoder was
the access protocol. A {\em synchronous} memory relies upon the access being completed a certain time after it begins, whereas an {\em asynchronous} memory requires the computer to wait for an acknowledgement signal.
I decided to implement  synchronous  memory, which was simpler but did rely upon the speed of memory being known.  The fifteen 
address bits and a {\bf r/$\overline{w}$} signal are used to indicate the location and 
direction of the access. 
A further signal, {\bf memory select} ({\bf ms}) is used to 
actually request a memory access. 

To write to an address the computer loads the {\bf address} register with the address 
and enables the output on the {\bf data} register. When both are known to 
be valid  {\bf ms} is asserted with {\bf r/$\overline{w}$} low, and decode 
begins. The signals must remain until it can be guaranteed that the 
write will be completed.
 This means  a write operation must include a delay  to allow for ALU operations.

\begin{figure}
\vspace{20cm}
\caption{Memory Access Protocols}
\label{fig:memread}
\end{figure}


To read an address the protocol is similar except the {\bf r/$\overline{w}$} line is set high 
and the {\bf data} register is not output onto the bus;  at the end of the read 
either the {\bf data} or {\bf instruction} register is loaded with the value on the 
data bus.
\subsection{Address Decode Implementation}

Most addresses are  decoded using EPLDS, rather than PALS. This 
is slower but provides for  later expansion of memory.
 One EPLD is  used 
for decoding the simple signals which can be asserted throughout the 
cycle ---the loading of   Execution Unit registers, and the reading of the Accumulator and  condition flags.

A further EPLD is  used for the generation of the one-off signals which    
load the {\bf ACC} and {\bf CC} registers after a function evaluation. 
It also controls the source of the count signal of the {\bf PC}.
 This EPLD 
is clocked to enable it to keep track of time during a write operation, and  contains a finite state machine.

The time to access the Random Access Memory is longer than for  other units, so in order to eliminate unacceptable decoding delays, the $\overline{write}$ and $\overline{Oe}$ signals for this are generated using hard wired logic. This  has a propagation delay of under ten nano-seconds.

To ensure there are no glitches during the decode, I found that the direction of the access must be changed prior to the assertion of {\bf ms}. Problems were caused when the control unit EPLD generated some outputs before others.

\section{Random Access Memory}

The  area of Random Access Memory  needed to be  fast and easy to use. 
This is why I decided from the outset to use Static RAM rather than Dynamic RAM, which while cheaper and denser was slower and much harder to interface to.
Deciding how much Static RAM to use, and at what speed  was a major source of problems,  due to cost and 
availability considerations.  
I had  originally envisaged that the computer would 
have a full 128kB memory from four 32k*8 Static RAM ICs. 
To increase the  projected speed of the computer I decided to  
 use very fast memories, but  being more expensive I had to have a 
 smaller amount.
 The suppliers R\&R advertised 16k*4 SRAMS with 25~nS access time, eight of which would have been ideal, 
especially with a cost of only \pounds 5.50 each. After much telephoning it 
became clear they did not have any in stock.  Another supplier did claim to have these memories, but 
\pounds 15 each was too expensive. 

Rather than rely on promises of availability I based my design on 8k*8 
memories of 45~nS.  
Four of these provide 8192 words,
which  places a tight limit upon the size of code which can be executed.


