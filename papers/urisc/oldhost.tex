\chapter{Host Interface}

The Ultimate RISC has to be connected to a host computer to obtain 
programs and data, to receive control signals and to return 
results. The host is able to examine and modify memory 
locations and the internal registers.

The traditional CS4 project method of building new computer systems 
is to use the standard APM interface bus to provide access to shared
memory . The disadvantage with this is that the APM 
bus is very difficult to interface to, and not all projects manage it 
successfully, even when using a known CPU or display controller. 
 To try and interface a CPU which could not be guaranteed to work reliably was infeasible.
The method I  use  allows the host 
computer to examine and modify both memory and internal 
registers on the Ultimate RISC computer. 
By making  all internal registers out of Serial Shadow Registers,
 the host can easily read and write to them. A number of control 
signals are needed, to copy all registers to their shadows, to 
shift data and to selectively reload registers.
The host also has ability to control the clock signals of the URISC.
This clock qualification enables slow stepping of the system, simplifying both
hardware and software development.

 Memory cannot be read or written directly. Instead one 
must initialise the {\bf address} register with the desired address --along with  the {\bf data} register during a write operation. The control unit can then be instructed
to perform the appropriate access. After a read operation the {\bf data} register
should contain the contents of the memory location selected.

Six bits of  state information are sent in parallel from the Ultimate RISC to the APM.
Five of these bits give the current state of the control unit. 
The sixth bit is sent from the memory section to indicate whether the current memory access is valid or not.
This enables the state of the machine to be 
determined without interrupting it.

Three signals are sent to the Ultimate RISC's control unit by the APM.
These can instruct it to start and stop executing instructions, or to perform a memory access for the host.

The complete list of connections is given in  table~\ref{table:hostif}, along 
with their position the eurocard edge connector.

The interface requirements are basic enough that it should be 
connectable to any host with a Peripheral Interface Adaptor (PIA) interface --which has 16  lines for external input and output.

The Real Time Systems M6809 co-processor board does have one of these devices, and
it is to this that my board is designed to connect. It can in fact be connected to many simple computers, provided the software support is available.

The host must execute a monitor program to communicate with the Ultimate RISC. This is described in a later chapter.

\begin{table}
\label{table:hostif}
\begin{tabular}{||l|lll||}
\hline
direction & signal & pin no. & description \\
\hline 
Output
 & State 0 & C2 & these give the current state of the control unit\\
 & State 1 & C3 & \\
 & State 2 & C4 & \\
 & State 3 & C5 & \\
 & State 4 & C6 & \\
 & $\overline{halt}$	& C7 & indicates the current read should cause a Halt\\
 & SDI	& C8 & SSR chain input to the host\\
\hline
Input 
 & SDO & C7 & SSR chain output from the host\\
 & $\overline{freerun}$ & C12 & instructs the URISC to run from its own clock\\
 & clock & C13 & the clock signal when not running freely\\
 & $\overline{go}$ & C14 & execute an instruction or a memory access\\
 & $\overline{load}$ & C15 & load {\bf address} and {\bf data} registers\\
 & l/$\overline{s}$ & C16 & indicates the type of the memory access\\
 & mode & C17 & SSR control\\
 & dclk & C18 & SSR shadow register clock\\
 & $\overline{reset}$ & C19 & reset program counter\\
 \hline
 \end{tabular}
 \caption{Host Interface Specification}

 \end{table}
 
 

