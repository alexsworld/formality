\chapter{The Host Interface}

The Ultimate RISC has to be connected to a host computer to obtain 
 data, to receive control signals and to return 
results. This host is able to examine and modify memory 
locations and the internal registers.

The obvious choice for a host computer was the departmental Advanced Personal Machine (APM) workstation.
These were widely available for use and designed to permit extra boards to be easily installed.
The compiler was also being developed upon this system, so the two machines could form a standalone system.


The traditional CS4 project method of connecting new boards to the APM
is to use the standard APM backplane to provide access to shared
memory. The disadvantage with this is that the APM 
bus is very difficult to interface to.
Not all projects manage it 
successfully, even when using a known CPU or display controller. 
 To try and interface a computer which could not be guaranteed to work reliably would have been futile.
The method I  used  allows the host 
computer to examine and modify both memory and internal 
registers on the Ultimate RISC computer. 
By making  internal registers out of Serial Shadow Registers,
 the host  can easily access them. 
Control 
signals are used to copy all registers to their shadows, to 
shift data and to selectively reload registers.
The host  also controls the clock signals of the computer, aiding  both
hardware and software development.

 Memory cannot be read or written directly. Instead the host 
 initialises the {\bf address} register with the desired address ---along with  the {\bf data} register during a write operation. 
 The control unit is then  instructed
to perform the appropriate access. 
After a read operation the {\bf data} register
should contain the contents of the memory location selected.

Six bits of  state information are sent in parallel from the Ultimate RISC to the APM.
Five of these bits give the current state of the control unit. 
The sixth bit is sent from the memory section to indicate whether the current memory access is valid or not.
The Ultimate RISC can thus be observed while operating asynchronously with respect to the host.

Three signals are sent to the Ultimate RISC's Control Unit by the APM.
These  instruct it to start and stop executing instructions, or to perform a memory access for the host.

The complete list of connections is given in  table~\ref{table:hostif}, along 
with their positions on the Eurocard edge connector.

In all, sixteen control lines are needed ---seven inputs to the host and nine outputs.
This allows the Ultimate RISC to be connected to any host with a sixteen bit parallel I/O port.

The Real Time Systems M6809 co-processor board does have one of these devices, and
it is to this that the board is  connected. It can in fact be connected to many simple computers, provided the software support is available.

\begin{table}
\centering
\begin{tabular}{||l|lll||}
\hline
direction & signal & pin no. & description \\
\hline 
Output
 & State 0 & C2 & these give the current state of the control unit\\
 & State 1 & C3 & as a five bit number\\
 & State 2 & C4 & \\
 & State 3 & C5 & \\
 & State 4 & C6 & \\
 & $\overline{halt}$	& C7 & indicates the current read should cause a Halt\\
 & SDI	& C8 & SSR chain input to the host\\
\hline
Input 
 & SDO & C7 & SSR chain output from the host\\
 & $\overline{freerun}$ & C12 & instructs the computer to use its own clock\\
 & clock & C13 & the clock signal when not running freely\\
 & $\overline{go}$ & C14 & execute an instruction or a memory access\\
 & $\overline{load}$ & C15 & load {\bf address} and {\bf data} registers\\
 & l/$\overline{s}$ & C16 & indicates the type of the memory access\\
 & mode & C17 & SSR mode selection \\
 & dclk & C18 & SSR shadow register clock\\
 & $\overline{reset}$ & C19 & reset program counter\\
 \hline
 \end{tabular}
 \caption{Host Interface Specification}
\label{table:hostif}
 \end{table}
 
 

\section{The Monitor}

The simplicity of the host interface is such that a sophisticated monitor 
program is needed to make any use of the Ultimate RISC.
The development of this  program went hand in hand with the building 
of the computer itself.
When a new feature was built into the hardware, it was first tested with the
existing monitor features, and, once the protocol was established, supported
by higher level routines.

{\samepage
The facilities offered include:-
\begin{itemize}
\item downloading and execution of programs on the 6809 board.
\item manual manipulation of the  interface lines.
\item reading, modifying and copying back the registers of the Ultimate RISC.
\item single stepping the computer's clock.
\item reading and writing memory locations ---including registers
\item memory testing
\item downloading code into the Ultimate RISC's memory
\item instruction execution control
\item emulated I/O on behalf of the Ultimate RISC.
\end{itemize}
}

The monitor is divided into two programs, each executing 
concurrently on separate processors and communicating  via shared memory.

\subsection{A Virtual PIA}
The Ultimate RISC's host interface is  connected to the Peripheral Interface
Adaptor of the APM 6809 Processor Board. While most of the memory is shared 
with the 68000 processor this is not the case for the I/O devices.
A short 6809 Assembly Language program is used to provide a virtual PIA for 
the 68000 microprocessor to use. 
This copies the input port of 
the PIA to a shared memory location, and uses data in other
memory locations to update the PIA registers.
This PIA contains two  eight bit I/O registers.
Port~A of the PIA contains all the outputs from the Ultimate RISC, and the SDO signal. 
Port~B is dedicated to the output of the eight control signals. 
This port is buffered to increase its drive capability.
The buffer used on most of the 6809 boards is an LS TTL device, but fanout effects  forced  me to upgrade the buffer on the board I was using to one of higher power. 
 

The first few memory locations of the 6809 processor's address space are dedicated to this inter-processor communication (table~\ref{table:shared}).

\begin{table}
\centering
\begin{tabular}{||r|l||}
\hline
\hline
Address & Description\\
\hline
0 & the PIA register (0--3) to write to\\
1 & the data to be written\\
2 & the inputs to PIA port A\\
\hline
4 &   loop count to drive Port B; cleared on completion of loop\\
8 &  port B loop value \#1\\
9 & port B loop value \#2\\
\hline
1E & the number of shared registers\\
1F & copy control =\$80 to get the registers, \$81 to write back\\
20 & the start of the shared SSR image\\
\hline
\hline
\end{tabular}
\caption{6809 Shared Memory Locations}
\label{table:shared}
\end{table}

When writing to a port of the PIA,
the 68000 program  writes the data first, followed by the register number plus 128;
this high bit  indicates a  valid write request.
When the contents of address 1 are cleared, the request has been processed and the copy of port A updated. The 68000 can use this to poll for the completion of an operation or test the operation of the 6809 board.
This method of synchronisation in not particularly fast, so
when controlling the clock from the host, the maximum clock speed is only 1~KHz.

To increase speed the 6809 program  contains  some extra routines. These perform iterative operations without the need for constant synchronisation between the Virtual PIA and the main processor.
The host's image of the Ultimate RISC's registers are stored in shared memory.
The PIA program can be instructed to shift in a new copy, or shift out an updated copy.
It can also be instructed to send an alternating pattern of signals out through port B a set number of times. 
This can be used to provide a faster clock signal of 50~KHz.

\subsection{The Main Monitor}
This is an extension of the CS2 6809 monitor, from which the Command Line Interface and the 6809 control operations are all taken.
It has been extended to form a 2000 line IMP program with many more commands.
There is also a status display at the top of the screen.
This shows the state of all the inputs and outputs on the host interface and 
the state of the control unit  as a mnemonic name.

To provide flexibility in development  the  data about registers and states are kept in separate files. 
This allows different configurations of both the control unit and of the registers to  be used with minimal effort.


Object code can be downloaded into the Ultimate RISC's memory by the monitor.
The format was designed to support both  compiled and hand coded programs
and is described in table~\ref{table:format}.
It is when downloading code that the speed of the host interface becomes apparent;
the liberal insertion of comments  provides needed feedback as to the progress of the  operation.

\begin{table}
\begin{enumerate}
\item the file is stored as a text file, with normal restrictions on naming
\item all numbers are given in hexadecimal, one to a line
\item any line beginning with `{\bf !}' is a comment, to be printed during
downloading
\item @(address) states the shared printing address (default=\$3FFF)
\item code sequences are represented as:-\\
address\\
code length\\
(code)$^{*}$
\item there is no limit upon the number of code sequences in a single file
\end{enumerate}
\caption{Code File Format}
\label{table:format}
\end{table}

The monitor program also provides a rudimentary output facility for the Ultimate RISC.
A downloaded program can specify a shared printing address.
Whenever this program halts, the instruction which caused the halt is examined.
If it is the instruction `{\bf MOVE~0000,0001}' then this is interpreted as a request for output.
The contents of the shared printing address are then printed on the terminal screen as an ASCII string, and the program is restarted.
%A similar  halt `{\bf MOVE~0000,0002}' causes the monitor to request a character from
%the user. 


The monitor is  somewhat isolated from the Ultimate RISC,
as the virtual PIA and the sixteen bit I/O port form a bottleneck in communication.
The monitor's registers  are only copies of the actual registers, and have to be updated explicitly, while
the status line is of more use to hardware development than  software.
However, the monitor does provide  adequate access to the Ultimate RISC.


