\chapter{The Formal Specification}


\section{Methodology}
The computer has been formally specified at the architectural level.
 The specification methodology used was {\bf Lambda} \cite{ahl:lambda}.
Designed especially 
 for hardware specification, it is implemented as an ML 
based system upon the Sun workstations. 
A semi-automated theorem prover  forms the 
core of Lambda, which enables properties of specifications to be 
examined, and reifications to be verified.

The system is designed to synthesise a working design from a 
behavioural specification, by reifying the design until it 
describes individual components ---{\em forward synthesis}.
 In such a way the correctness of a design 
can be guaranteed without performing a verification. 
Written in the language ML it has a similar syntax, but with extensions to the language.
This enables part of a specification to be executed as  ML functions,
allowing the hardware's  properties to be simulated in software.

The  Lambda specification language is more powerful than an 
executable programming language. Rather than describing a function 
or procedure to convert from the input to the output, one just specifies  preconditions and
postconditions. 

{\samepage
For example, the function:-
{\tt
\begin{verbatim}
        fun square_root (x:Natural) = iota y. y*y==x;
\end{verbatim}
}

describes the square root function in Lambda but not ML.
}
Timing constraints can also be included into {\em rewrite rules}
{\tt
\begin{verbatim}
        val sqrt_unit#(x,y)=
                forall t.
                     y (t+100) == square_root (x t);
\end{verbatim}			
}
This describes a combinatorial square root unit which outputs at time
t+100 the square root of the input at time t.

Functional units can  be joined together by use of common variable bindings
{\tt
\begin{verbatim}
        val double_sqrt_unit#(x,z)=
              sqrt_unit(x,y) /\ sqrt_unit(y,z);		
\end{verbatim}
}
Such functions, along with the definition of the types of the variables of X,Y and Z, can be parsed by  Lambda to produce an environment of rules.
These rules can be used to prove hypotheses, such as that the time for a double\_sqrt\_unit to evaluate an expression would take 200 units of time.


This system allows someone to specify any component  or module as a collection of related inputs and outputs.
A number of components can  be linked together to form a larger module,
or a complete design.
	
%\newpage
Unfortunately, none of  the components I used had  
been formally specified. While I  produced some specifications based on the informal specifications in the databooks their correctness  can never be guaranteed. 


Specification from scratch is an extremely slow process, and I was not able to describe the whole computer in this depth in the time available.
 I  described the operation of the computer at a very high level, and then  expanded the description of the ALU to a greater depth.

\section {First Specification} 

My first specification was based upon an example specification of a simple computer in the Lambda Manual.
It was written in a early version of Lambda, which had a syntax more complex than that of ML.
First it was necessary to specify the types of data which the computer dealt with. 
Abstract datatypes of 15 and 32 bit integers were defined without giving their internal structure.
Allowable 
operations ---comparison and addition for 15 bit numbers, all ALU 
operations for 32 bit numbers--- were stated as existing and being 
total. Their exact functions were not given.
 Random Access Memory 
was then defined as a function of 15 bit addresses to  32 bit data words. 

The computer can at any moment in time  be described by  the contents of every  register
and  memory location.
 As an instruction is executed this state   changes,
unless it is halted or in an infinite loop.

The state of the 
computer was described as a tuple of
\begin{verbatim}
        <memory, execution unit state, alu state> 
\end{verbatim}
where the execution unit state was 
\begin{verbatim}
        <PC,X,Y,skip,halt>
\end{verbatim}
    and the
ALU state was
\begin{verbatim}
        <ACC,z,n,v,c>
\end{verbatim}
 

 The operation of the whole machine was  given as a transition from 
state to state. 
Each transition was caused by the  execution of
 a single instruction.
The most complex part of this specification was the description of the read and write operations.
This was because of the memory mapping of registers.
The read function was supplied with the computer state and a 15 bit address to return a 32 bit integer. 
A separate function was used to validate the address prior to the read access.
The write function took as parameters a  state, an address and a new value, returning a new state. This allowed both registers and RAM to be updated.

The transition from one state to another during instruction execution was described by a function which  fetched the next instruction, and incremented the PC.
It then calculated the source  address and moved its contents to the  destination address.  
If the halt flag was set this function did nothing. 
Before each read the address was validated ---any illegal access terminated the function and set the halt flag.
If the skip flag was set the program counter was merely incremented and the flag cleared; no instruction was executed.

This does actually resemble the actual process of instruction fetch
and execute, except that the halt and skip flags do not actually
exist. 
Skipping is performed by hard-wired logic, and the halt state is merely another state within the control unit's Moore Machine.
These differences are invisible to the user. 
At this level the operation of the computer is being described, even if it differs slightly from the actual implementation.


\section{Second Specification}

By Christmas a new version of Lambda was available, with a syntax more  similar to ML. The design of the Ultimate RISC had become clearer, partly through the initial specification, but also as I  designed the ALU.

I therefore upgraded the specification to support both the new notation and to be consistent with the revised design.

This was done by first expanding the 15-bit and 32-bit integer abstract data types to  boolean tuples, with functions for conversion between these representations and that of natural numbers.

I then wrote all the operations performed by a 74381 ALU IC as 
functions acting upon boolean four-tuples. A general function was 
written to apply the operation selected by the control lines.
 The production of carry signals from the 74182 carry lookahead generators were 
also specified. 
In both cases the specifications were based upon  
data sheets from TI and AMD. 
It was then possible to describe the ALU components by relating outputs as 
the result of the functions applied upon the inputs of a previous time 
---the temporal difference being the propagation delay of the device.

The logic equations of the PALS were all specified likewise, enabling the 
entire combinatorial portion of the ALU to be accurately described.

The remainder of the specification was derived from the first  specification of the Ultimate RISC. 
 
I did not go about proving the two specifications were identical.  
Nor did I 
 try to prove properties of the new specification, such as
the non-folding of 
RAM addresses and the persistence of data within. 
 Given estimates of 
the time to verify the correctness of specifications of other computers, 
there would have been no possibility of both verifying the specifications and 
attempting to build anything.
The verification would probably  form a complete project, 
requiring someone far more experienced in machine assisted proofs than myself.
Instead I  produced a simulation, by modifying the specification  to  execute in ML.

 To enable my 
specification to be executed I had to remove all instances of 
postconditions and timing constraints. 
The other problem was `feedback'.
The 74381 units produce signals which are 
passed to the carry generators. These then  return a carry signals to be evaluated with the earlier inputs. 
These were simulated by iterations of the functions.

I also wrote a `monitor' for the simplified specification which provided a 
front end for the simulation with facilities such as memory read/write, 
instruction assembly and dissassembly, register manipulation and 
program execution. 

There were  subtle differences between ML and Lambda. Notably ML's integers were more restricted than Lambda's type Natural.
 Both the Edinburgh SML and the faster New Jersey ML had only 32 bit signed integers, rendering conversion between these numbers and 32 bit boolean tuples difficult.
The specification only executed satisfactorily in PolyML,
which was   prone to unannounced field upgrades, so that functions   available in February did not work in April.



\section{Summary}

Overall the specification and simulation comprise 
 1400 lines, and are somewhat more difficult to understand at a 
glance than P-CAD circuit diagrams. Much of the code is devoted to 
mathematical and component definitions. For the success of hardware 
specification languages I believe it will be necessary to produce 
libraries of verified maths functions, components  and standard cells.

 It is not a complete specification of the hardware of the 
 Ultimate RISC.
 For this the computer must be described as a collection 
of units, communicating via control signals and synchronised by a 
clock. 
The time delays of all operations must be specified, along 
with the ability of the host to control the clock and write to 
registers. 

The simulation has been used to test the operation of the ALU, 
especially the programming of the PALS,  uncovering a couple 
of mistakes which could have proved  costly.
It also demonstrated that the carry flag had to be set prior to a subtraction,  a fact the compiler writer needed. 

While I have not performed any proofs, specifying the computer was a valuable exercise. 
 Describing the machine at a high level, I was forced to consider many details which any other method of description would ignore;
 this clarified the hardware implementation. 

The specification of an existing design did seem easier than the forward synthesis method for which Lambda aims to provide. 
Forward synthesis should reduce the number of proofs required, and thus increase the speed of designing with formal methods.

Even without the verification between levels, a specification which describes the computer is of use  describing the system to a software developer. 
For this to be the case the software designers  need to be able to understand the notation. 
This is an argument in favour of using a more widely known notation such as {\bf Z}, which seems to be primarily for software development. 
If programs were specified in the same notation as the hardware  the two would be able to be integrated in order to prove facts about the combined system.

Ideally the specification should have been continued till every component was specified along with the interconnections.
A netlist could have been extracted and sent to the BEPI machine for automated wiring, and the PAL and EPLD programs also generated.
If these processes could be at least partially automated, then the Lambda system could form the core of an an automated design and manufacture system.
Without such a system,   manual intervention ---whether P-Cad design or wire-wrapping--- introduces elements of risk. For the full benefit of formal methods the implementation must match the specification.






