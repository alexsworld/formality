\chapter{The ALU}

\section{Design}
For a computer to be capable of effective operation, it needs the ability to perform
 actual processing of data. 
It has long been known that the ability to compare two items and act upon the result is sufficient for effective computation 
---Turing Machines are based around this concept.
It would therefore have been possible to build a basic  comparison unit, and rely
on software  to derive mathematical and logical operations.
This would have been unreasonably  inefficient.
All realistic computers   have hardware dedicated to evaluation of these functions.
These Arithmetic and Logic Units ({\bf ALU}) normally perform at least integer addition, subtraction and the standard boolean functions of two variables.
More powerful units are capable of high speed multiplication, or even manipulate
floating point numbers.


At the start of the project I was offered the possibility of using a 
single chip 64-bit floating point ALU from AMD (AM29C327) \cite{amd:uprogramming,amd:29c300}.
This would have produced impressive performance figures, but I 
decided that it would have been unworkable, since it was 
designed for a triple data bus and needed 31 bits of control 
information every cycle. A single bus system would have  been unable to use this 
device effectively. 

Instead I designed a very simple ALU, since this made 
formal specification  possible. The unit was  
built from eight bit sliced TTL ICs, each of which  operates on four bits.
 When connected together via a two level 
 carry lookahead generator, they  perform   operations on 32-bit words.

This 
is sufficient for many purposes, except that the ability to shift 
a word  right was needed in iterative multiplication and division algorithms.

 The result of the ALU had to be stored until  re-used in later instructions.
The state of this result, 
whether zero or  negative  needed to obtained in a form
 which could be passed to the Skip register.
 Arithmetic overflow and carry flags were also desirable,
   detecting results too large to be represented in 32 bits.
 
\section{Implementation}

The design of the ALU is shown in figure~\ref{figure:alu}.

An Accumulator stores the output of the ALU between operations.
This can be read as a memory location.
The contents of the Accumulator are also used as one of the inputs to the ALU,
so only one other argument needs to be supplied per operation.
This accumulator is built out of four SSRS, so can be read directly by the host.

A number of bit sliced ALUs were available with built in accumulator registers. 
For example,
the AMD AM2901 (\cite{amd:logic}) or the TTL 74F681 ALU bit slices,
 would have provided enhanced performance with less components and wiring.
Using these would have prevented the host examining the Accumulator directly.
Instead I used 74F381 ALU/function generators in my design.
These  only perform basic operations ---addition, subtraction, and, or, exclusive or, preset and clear. 
Three control signals  are used to select a function.


Between the outputs of the ALU ICs and the Accumulator is a bank of five PALS.
Normally these pass the result straight through, each PAL checking if the bits passed though it are all zero or not.
They can also be instructed to shift the result ---including the carry flag--- one bit to the right; 
this shifting is controlled by a  one bit signal.
This post shifting  allows a normal operation to be combined with a shift, to make unusual functions such as `subtract and divide by two'.

The results of the five 
zero tests along with other signals are fed to another PAL, which 
produces values for a Condition Code register ({\bf CC}),  constructed from a Shadow Serial Register. 
The PAL generates a zero flag  when all five slices of the result are zero.

\begin{figure}
\vspace{20cm}
\caption{The ALU}
\label{figure:alu}
\end{figure}

\subsubsection{Overflow}

An arithmetic overflow is where  a signed number's sign changes due to too large an addition, subtraction or shift.

My design of an ALU does not detect signed overflow, despite the original intent to do so.
I had originally
acquired equations  from my CS3 notes to detect  overflows using a PAL.
While specifying the system  I realised
these equations  only detected overflow on signed addition. 
To detect overflow in a multi-function ALU, one must compare the carry between
bit 30 and bit 31 of the result with the most significant bit, an overflow occuring if the two differ.
This can not be done with the 74F381 bit-sliced devices, as this carry is internal.
I have discovered that AMD make a special most-significant-slice version of this bit-sliced ALU which does detect overflows internally. 
The result of this check would however become confused if shifting was performed after the operation, so would not always be reliable.

 Note that even if the ALU did produce an overflow flag, the software would still have to check it after every operation. A number of
implementations of languages do not do this because of the overhead this entails; 
APM Pascal and Standard ML are two such  implementations.

\subsubsection{Memory Interface}

Seventeen addresses are allocated to the ALU, as shown in table~\ref{table:memory}.
One of these addresses returns the current value of the Accumulator whenever it is read.
The remaining sixteen addresses all apply a different function between the accumulator and the word moved to the selected address.
This is accomplished by wiring  address bus lines directly to the ALU  and the PALS.

It is not be possible to directly load the 
accumulator, but a two instruction sequence  clears it and then adds a 
number to the now empty accumulator.

Condition code flag manipulation is  supported:
reading any of the sixteen function addresses returns one of the condition code flags in the least significant bit. 
These results can be passed directly to the Skip register for conditional branching.
Before performing a subtraction the carry flag has to be set to true, while
for other operations the flag has to be cleared.
An address is provided to enable this; when it is written to, the least significant bit is passed to the carry flag.
