\chapter{Implementation Issues}

Throughout the project I had always to bear in mind a number of issues:-
\begin{itemize}
\item 
The computer must be interfaced to an APM, and so  built on a double size Eurocard.
\item
The budget for the project could not exceed \pounds 300
\item
The computer had to be designed, built and documented by late May.
\item
The choice of manual or automatic board construction.
\end{itemize}
These  resulted in a computer which lacks some of the features which are  normally expected.

\section{Components}


It was impossible to build my computer out of a few standard off-the-shelf VLSI components; it bore too little resemblance to existing systems. Fortunately, a large number of MSI components for building computers have long been available, and I  made use of these.
Individually quite small, together they  occupy a large amount of area  and have an excessive power consumption. 
It is worthwhile describing the main types of components here, rather than repeatedly describe them later.

\subsection{Transistor-Transistor Logic (TTL)}
TTL integrated circuits have been the standard building blocks for digital circuits for over a decade. They have a number of advantages:-
\begin{itemize}
\item widely available
\item individually inexpensive
\item simple enough to be  flexible in application
\item well documented
\item  operate at high speed
\end{itemize}
 Many of these  components need to be wired together to produce a useful circuit. Some types of TTL `families' are low-power versions, being slow and suffering from fanout problems. I decided to use 74F (FAST) series logic, which were high power and high speed. 


\subsection{Programmable Array Logic (PALS)}

These are integrated circuits which can be programmed with simple boolean logic equations expressed in the sum of products form \cite{mmi:pals}. 
They were used in situations where a standard component was not available. 
The fact that they were explicitly programmed  made them correspond well to the formal specification, the equations actually being edited
from the specification to the PAL programming format. 
This was important as these devices could not be reprogrammed ---first time correctness had to be guaranteed.

\subsection{Erasable Programmable Logic Devices (EPLDS)}

These are a development of PALS, being erasable in ultra-violet light.
 This made them useful in the parts of the computer which needed reprogramming during development. 
They were much more expensive, and did not operate as fast as PALS do. 
The extra capital cost can be recuperated  by re-use in later applications.


\subsection{Serial Shadow Registers (SSRS)}
These eight bit registers, manufactured by AMD, formed a central part of my design \cite{amd:logic,amd:uprogramming}. 
Designed for use in pipelined computers, they  behaved as   eight bit tristate registers. 
They also had extra  circuitry designed to aid debugging. Each register had a shadow register which was invisible to normal operation. 
Each device could be instructed to copy the state of the outputs into this shadow, or overwrite the main register with the contents of the shadow. 
This shadow register could be shifted out one bit at a time, while  new bits were shifted in. 
 A number of these registers were connected together  via these serial links to form a chain.
 By connecting the register chain to the host computer, this computer was able to read and write  the shadow registers.
 This permitted the host to  discover the state of the Ultimate RISC, and to indirectly access its memory.
In some respects the use of these SSRS produced a design which was a bit less efficient than would otherwise be possible, but this is made up for by the observability of the system.
For example,  the ALU could have been implemented in a third of the number of ICs, but would then have been observable only indirectly. 

\section{Wiring Methods}

I had originally planned to have the   computer automatically  wired using the Department's BEPI solderwrap machine. 
This  takes a circuit board with capacitors, IC sockets, and a list of the wiring between the pins,  then wires up all the interconnections. 
The choice of a 32 bit data bus was made on the assumption that I would be using this system and so the extra wiring would not cause any construction delays.


As time passed I became aware of the disadvantages to prototyping on the BEPI machine.
The main problem with  solderwrapping  was that it was impossible to modify a board once built, and extremely difficult to debug if faulty. 
It may have been  faster to build than wrapping by hand, but there was the effort to be  spent actually generating the files needed by the BEPI 
machine.
What is more, there were even some doubts as to the correctness of the wiring which the solderwrap machine produced.

Taking these factors into account I eventually decided to wire up my computer
manually. This allowed incremental development of the computer, which 
meant I was always  able to demonstrate some part of the computer 
working, rather than just a finely wired but useless board.

